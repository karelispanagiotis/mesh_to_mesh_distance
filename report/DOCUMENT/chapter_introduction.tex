\chapter{Εισαγωγή}
\label{ch:introduction}
\section{Κίνητρο}

\section{Περιγραφή του Προβλήματος}

\section{Στόχοι της Διπλωματικής Εργασίας}

\section{Διάρθρωση της Διπλωματικής Εργασίας}
Στο \textbf{Κεφάλαιο \ref{ch:introduction}} έγινε μια εισαγωγή στο 
πρόβλημα υπολογισμού της απόστασης δύο πλεγμάτων και παρουσιάστηκαν
τα κίνητρα που οδήγησαν στην υλοποίηση των αλγορίθμων που θα 
παρουσιαστούν. 

Στο \textbf{Κεφάλαιο \ref{ch:theoretical_background}} παρουσιάζεται το 
θεωρητικό υπόβαθρο που απαιτείται από τον αναγνώστη ώστε να κατανοήσει
πλήρως το πρόβλημα και την προτεινόμενη λύση.

Στο \textbf{Κεφάλαιο \ref{ch:related_work}} αναφέρεται η σχετική 
βιβλιογραφία, δηλαδή πώς αντιμετώπισαν άλλοι ερευνητές το ίδιο ή 
παρόμοια προβλήματα. 
Παρουσιάζονται επίσης ομοιότητες και διαφορές
των υπολοίπων προσεγγίσεων σε σχέση με τη δική μας.

Στο \textbf{Κεφάλαιο \ref{ch:methodology}} αναλύεται η μεθοδολογία 
που προτείνουμε και παρουσιάζεται η υλοποίηση των αλγορίθμων μας.

Στο \textbf{Κεφάλαιο \ref{ch:experiments}} παρατίθενται τα αποτελέσματα 
από τα πειράματα που εκτελέσαμε. 
Οι μετρήσεις των πειραμάτων περιλαμβάνουν την εκτίμηση της συνάρτησης 
κόστους που ορίζεται στην ενότητα \ref{sec:cost_metric} καθώς και τους χρόνους
εκτέλεσης των αλγορίθμων. 

Στο \textbf{Κεφάλαιο \ref{ch:future_work}} σχολιάζονται τα αποτελέσματα
και παρουσιάζονται σκέψεις για μελλοντική επέκταση και βελτίωση των 
ιδεών της παρούσας διπλωματικής εργασίας.

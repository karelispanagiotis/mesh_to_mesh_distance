\chapter{Εισαγωγή}
\label{ch:introduction}
\section{Κίνητρο}
Ο υπολογισμός της απόστασης που διαχωρίζει δύο αντικείμενα στο 
χώρο αποτελεί θεμελιώδες πρόβλημα στον τομέα της Υπολογιστικής
Γεωμετρίας.
Στη ρομποτική, στη σχεδίαση και μηχανική με τη βοήθεια υπολογιστών 
(\tl{CAD} και \tl{CAE}), στις προσομοιώσεις με υπολογιστές και στη γραφική 
υπολογιστών είναι σημαντικό να γνωρίζουμε εάν δύο αντικείμενα, που περιγράφονται 
από μαθηματικά μοντέλα στον τρισδιάστατο χώρο, τέμνονται/συγκρούονται ή βρίσκονται 
σε κοντινή απόσταση.
Για την παρούσα διπλωματική εργασία, εξετάζουμε την περίπτωση όπου τα αντικείμενα
του χώρου περιγράφονται από τριγωνικά πλέγματα (βλ. \ref{sec:triangle_meshes}). 

Στη ρομποτική, για παράδειγμα, η επίλυση του παραπάνω προβλήματος 
είναι απαιτητή για τον σχεδιασμό διαδρομής με την παρουσία εμποδίων
\tl{(path-planning problem)}
\cite{brooks1985subdivision},
\cite{cameron1985study}, 
\cite{canny1986collision}, 
\cite{culley1986collision}.
Επιπλέον, σε εφαρμογές \tl{CAD-CAE} στις οποίες σχεδιάζονται περίπλοκες 
κατασκευές που αποτελούνται από μεγάλο αριθμό εξαρτημάτων, ο εντοπισμός 
σύγκρουσης μεταξύ των εξαρτημάτων είναι απαραίτητος τόσο για τις αναλύσεις 
και δοκιμές των προϊόντων όσο και για την παραγωγή τους 
\cite{boyse1979interference}. 
Στη γραφική με υπολογιστές και συγκεκριμένα κατά την κίνηση των 
αντικειμένων σε μια εικονική σκηνή, όπως στα βιντεοπαιχνίδια και 
στα κινούμενα σχέδια, είναι πιθανό τα αντικείμενα να διεισδύσουν το 
ένα στο άλλο. Αυτή η κατάσταση δεν είναι επιθυμητή όταν με τα 
γραφικά επιδιώκεται η αναπαράσταση ενός ρεαλιστικού κόσμου
\cite{moore1988collision}.
Τέλος, το πρόβλημα υπολογισμού της απόστασης που διαχωρίζει δύο αντικείμενα 
στο χώρο συναντάται και στον τομέα της υπολογιστικής φυσικής για εφαρμογές 
ανάλυσης πεπερασμένων στοιχείων (\tl{FEA}) και προσομοιώσεων 
\cite{khamayseh2007use}.

Η ραγδαία ανάπτυξη όλων των παραπάνω τομέων τα τελευταία χρόνια
και η απαίτηση λεπτομερέστερης περιγραφής των αντικειμένων του
τρισδιάστατου χώρου, κάνουν επιτακτική την ανάγκη για μελέτη
και σχεδιασμό αποδοτικών αλγορίθμων που μπορούν να διαχειριστούν
μεγάλους όγκους δεδομένων εισόδου. Σε αυτή τη διπλωματική εργασία
προτείνουμε αποδοτικούς αλγορίθμους για τον υπολογισμό της απόστασης 
δύο τριγωνικών πλεγμάτων στον τρισδιάστατο χώρο.


\section{Περιγραφή του Προβλήματος}

\section{Στόχοι της Διπλωματικής Εργασίας}

\section{Διάρθρωση της Διπλωματικής Εργασίας}
Στο \textbf{Κεφάλαιο \ref{ch:introduction}} έγινε μια εισαγωγή στο 
πρόβλημα υπολογισμού της απόστασης δύο πλεγμάτων και παρουσιάστηκαν
τα κίνητρα που οδήγησαν στην υλοποίηση των αλγορίθμων που θα 
παρουσιαστούν. 

Στο \textbf{Κεφάλαιο \ref{ch:theoretical_background}} παρουσιάζεται το 
θεωρητικό υπόβαθρο που απαιτείται από τον αναγνώστη ώστε να κατανοήσει
πλήρως το πρόβλημα και την προτεινόμενη λύση.

Στο \textbf{Κεφάλαιο \ref{ch:related_work}} αναφέρεται η σχετική 
βιβλιογραφία, δηλαδή πώς αντιμετώπισαν άλλοι ερευνητές το ίδιο ή 
παρόμοια προβλήματα. 
Παρουσιάζονται επίσης ομοιότητες και διαφορές
των υπολοίπων προσεγγίσεων σε σχέση με τη δική μας.

Στο \textbf{Κεφάλαιο \ref{ch:methodology}} αναλύεται η μεθοδολογία 
που προτείνουμε και παρουσιάζεται η υλοποίηση των αλγορίθμων μας.

Στο \textbf{Κεφάλαιο \ref{ch:experiments}} παρατίθενται τα αποτελέσματα 
από τα πειράματα που εκτελέσαμε. 
Οι μετρήσεις των πειραμάτων περιλαμβάνουν την εκτίμηση της μετρικής 
κόστους που ορίζεται στην ενότητα \ref{sec:cost_metric} καθώς και τους χρόνους
εκτέλεσης των αλγορίθμων. 

Στο \textbf{Κεφάλαιο \ref{ch:future_work}} σχολιάζονται τα αποτελέσματα
και παρουσιάζονται σκέψεις για μελλοντική επέκταση και βελτίωση των 
ιδεών της παρούσας διπλωματικής εργασίας.

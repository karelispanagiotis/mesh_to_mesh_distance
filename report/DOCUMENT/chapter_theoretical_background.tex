\chapter{Θεωρητικό Υπόβαθρο}
\label{ch:theoretical_background}

\section{Πλέγματα}
\begin{wrapfigure}{r}{0.25\textwidth}
    \centering
    \includegraphics[width=0.25\textwidth]{Dolphin_triangle_mesh.png}
    \caption[Παράδειγμα Τριγωνικού Πλέγματος]{
        Παράδειγμα τριγωνικού πλέγματος που αναπαριστά ένα δελφίνι.
    }
    \label{fig:example_mesh}
\end{wrapfigure}
\label{sec:triangle_meshes}
Στην Υπολογιστική Γεωμετρία τα \textit{πλέγματα} αποτελούν την αναπαράσταση
μιας μεγαλύτερης γεωμετρικής περιοχής από μικρότερα διακριτά στοιχεία.
Τα πλέγματα χρησιμοποιούνται συνήθως για τον υπολογισμό λύσεων μερικών 
διαφορικών εξισώσεων, για την απόδοση γραφικών υπολογιστών, και για
ανάλυση γεωγραφικών και χαρτογραφικών δεδομένων.
Ένα πλέγμα χωρίζει τον χώρο σε μικρότερα στοιχεία (πολύγωνα ή πολύεδρα) όπου
μπορούν να λυθούν οι εξισώσεις, το οποίο στη συνέχεια προσεγγίζει τη λύση 
στο ευρύτερο πεδίο. 
Τα πλέγματα που αποτελούνται από πολύεδρα αντιπροσωπεύουν ρητά τόσο την 
επιφάνεια όσο και τον όγκο ενός αντικειμένου, ενώ τα πολυγωνικά 
πλέγματα αντιπροσωπεύουν μόνο την επιφάνεια (ο όγκος υπονοείται).
Για το πρόβλημα υπολογισμού της Ευκλείδειας απόστασης, ενδιαφερόμαστε 
μόνο για την εξωτερική επιφάνεια των αντικειμένων του τρισδιάστατου χώρου. 

Ένας τύπος πολυγωνικών πλεγμάτων είναι τα τριγωνικά πλέγματα 
(σχήμα \ref{fig:example_mesh}).
Αποτελούνται από ένα σύνολο τριγώνων στις τρεις διαστάσεις, 
τα οποία συνδέονται με τις κοινές τους ακμές ή κορυφές. 
Γεωμετρικά, ένα πλέγμα είναι μια τμηματικά επίπεδη επιφάνεια.
Η τελευταία ιδιότητα ισχύει πάντοτε για τα τριγωνικά πλέγματα.

Υπάρχουν διάφοροι τρόποι για την αποθήκευση ενός τριγωνικού πλέγματος 
στη μνήμη του υπολογιστή. 
Υπάρχουν επίσης μέθοδοι μετατροπής του ενός 
τρόπου αποθήκευσης σε έναν άλλο.
Ενδεικτικά αναφέρουμε την αποθήκευση με:
\begin{itemize}
    \item \textbf{Σετ τριγώνων}: 
    Το πλέγμα αναπαρίσταται απλά από τα τρίγωνα 
    του. 
    Δηλαδή αποθηκεύονται οι συντεταγμένες των κορυφών κάθε τριγώνου.
    \item \textbf{Σετ Τριγώνων με δείκτες}: 
    Το πλέγμα αναπαρίσταται από ένα σετ
    κόμβων και ένα σετ από τριπλέτες με δείκτες στους κόμβους.
    Η κάθε τριπλέτα αναπαριστά ένα τρίγωνο.
    \item \textbf{Λωρίδες Τριγώνων}: 
    Η αποθήκευση αυτή βασίζεται στο γεγονός ότι 
    δύο γειτονικά τρίγωνα μοιράζονται τη μία τους πλευρά. Αυτός 
    ο τρόπος χρησιμοποιείται για συμπίεση των πλεγμάτων.
    \item \textbf{Δομή Τρίγωνου-Γείτονα}: 
    Υποστηρίζει ερωτήματα γειτνίασης τριγώνων.
    \item \textbf{Δομή \tl{Winged-Edge}}: 
    Αποθηκεύει δεδομένα κόμβων, ακμών και όψεων.
    Επιτρέπει εύκολη διάβαση στο πλέγμα μεταξύ όψεων, ακμών και κορυφών.
\end{itemize} 
Για τους αλγορίθμους που περιγράφουμε παρακάτω αρκεί ο πρώτος τρόπος 
αναπαράστασης. 

Τέλος, για τις διάφορες εφαρμογές που χρησιμοποιούνται, τα πλέγματα 
χαρακτηρίζονται και από την ποιότητα τους. Οι πιο συνηθισμένες μετρικές 
ποιότητας είναι:
\begin{itemize}
    \item \textbf{Λοξότητα (\tl{Skewness})}: \\
    Η λοξότητα είναι ο λόγος της απόκλισης μεταξύ του βέλτιστου μεγέθους
    στοιχείου προς στο υπάρχον μέγεθος στοιχείου. 
    Το εύρος της λοξότητας είναι μεταξύ 0 (ιδανικό) έως 1 (χειρότερο).
    Τα πολύ λοξά στοιχεία δεν προτιμώνται λόγω της κακής ακρίβειας 
    που προκαλούν στις παρεμβαλλόμενες περιοχές.
    Ανάλογα με το στοιχείο (τρίγωνο, τετράπλευρο, τετράεδρο, 
    εξάεδρο κλπ) διαφοροποιείται μαθηματικός τύπος 
    για τον υπολογισμό της λοξότητας.

    \item \textbf{Ομαλότητα (\tl{Smoothness})}: \\
    Η αλλαγή στο μέγεθος των στοιχείων πρέπει να είναι ομαλή. 
    Συνήθως αποφεύγονται ξαφνικά άλματα στο μέγεθος των στοιχείων γιατί αυτό 
    μπορεί να προκαλέσει λανθασμένα αποτελέσματα σε κοντινούς κόμβους.
    
    \item \textbf{Αναλογία Διαστάσεων (\tl{Aspect Ratio})}: \\
    Εν συντομία, ο λόγος διαστάσεων είναι ο λόγος του μεγαλύτερου μήκους 
    ενός στοιχείου προς το μικρότερο μήκος. 
    Ο ιδανικός λόγος διαστάσεων είναι 1. 
    Όσο μικρότερος είναι, τόσο υψηλότερη είναι η ποιότητα ενός στοιχείου. 
    Η μέθοδος υπολογισμού ποικίλλει ανάλογα με τον τύπο κελιού.
\end{itemize}

Σε πραγματικές εφαρμογές, τα πλέγματα συνήθως σχεδιάζονται έτσι ώστε 
να μην παραβιάζουν σε μεγάλο βαθμό τις παραπάνω μετρικές ποιότητας.
Αυτή η παρατήρηση είναι χρήσιμη για τον σχεδιασμό της δομής δεδομένων 
που προτείνουμε.

\begin{figure}[h]
    \centering
    \includegraphics[width=0.45\textwidth]{airplane_bad_mesh.png}
    \includegraphics[width=0.45\textwidth]{airplane_good_mesh.png}
    \caption[Παράδειγμα Ποιότητας Πλεγμάτων]{
        Παράδειγμα Ποιότητας Πλεγμάτων - Και τα δύο πλέγματα 
        αναπαριστούν το ίδιο αεροπλάνο. Το δεξί πλέγμα είναι 
        καλύτερης ποιότητας από το αριστερό που φαίνεται να 
        παραβιάζει όλα τα κριτήρια ποιότητας που αναφέρθηκαν
        (\tl{skweness, smoothness, aspect ratio}). Και τα 
        δύο πλέγματα αποτελούνται από τον ίδιο αριθμό τριγώνων
        περίπου (γύρω στα $15000$ τρίγωνα).
    }
\end{figure}

\section{Οριοθετικοί Όγκοι}
\textit{Οριοθετικός όγκος} (\textbf{\tl{bounding volume}}) ενός συνόλου 
από αντικείμενα του τρισδιάστατου χώρου ονομάζεται οποιοσδήποτε 
κλειστός όγκος που εξ' ολοκλήρου περικλείει τα αντικείμενα του συνόλου. 
Οι οριοθετικοί όγκοι χρησιμοποιούνται για να επιταχύνουν αλγορίθμους   
που εκτελούν γεωμετρικούς ελέγχους χρησιμοποιώντας απλούς όγκους 
που περικλείουν πολύπλοκα αντικείμενα.
Οι έλεγχοι σε οριοθετικούς όγκους είναι τυπικά πολύ ταχύτεροι από 
ελέγχους στο ίδιο το στοιχείο ή αντικείμενο που περικλείουν 
Σε πολλές περιπτώσεις, ένας έλεγχος αρκεί για να απορριφθούν ή 
να επιβεβαιωθούν πολλαπλοί έλεγχοι που θα απαιτούνταν για κάθε ένα 
από τα στοιχεία ξεχωριστά.

Οι οριοθετικοί όγκοι βρίσκουν ευρεία εφαρμογή στους παρακάτω τομείς:
\begin{itemize}
    \item \textbf{Ανίχνευση Ακτίνων (\tl{Ray Tracing})}:\\
    Oι οριοθετικοί όγκοι χρησιμοποιούνται σε ελέγχους τομής ακτίνων 
    με αντικείμενα και σε πολλούς αλγόριθμους απόδοσης γραφικών.
    Για παράδειγμα, εάν η ακτίνα ή το οπτικό πεδίο της κάμερας
    δεν τέμνει τον οριοθετικό όγκο, τότε δεν μπορεί να τέμνει ούτε 
    το αντικείμενο που περιέχεται μέσα. Έτσι αποφεύγονται οι 
    αντίστοιχοι έλεγχοι, οι οποίοι κοστίζουν υπολογιστικά.
    Όμοια, εάν το οπτικό πεδίο της κάμερας περιέχει εξ ολοκλήρου 
    τον οριοθετικό όγκο, το αντικείμενο, δηλαδή τα στοιχεία από τα
    οποία αποτελείται θα απεικονιστούν στην 
    οθόνη χωρίς περισσότερους ελέγχους. 
    
    \item \textbf{Ανίχνευση Σύγκρουσης (\tl{Collision Detection})}:\\
    Όμοια με πριν, όταν δύο οριοθετικοί όγκοι δε συγκρούονται/τέμνονται, 
    τότε ούτε και τα αντικείμενα που περικλείουν δεν μπορούν να συγκρούονται.
\end{itemize}

Για τη δημιουργία οριοθετικών όγκων σύνθετων αντικειμένων, συνήθως 
χρησιμοποιούνται \textit{Ιεραρχίες Οριοθετικών Όγκων} (βλ. \ref{sec:bvh}).
Δηλαδή, δενδρικές δομές δεδομένων όπου η βασική ιδέα κατασκευής τους 
είναι η ρίζα να περικλείει ολόκληρο το αντικείμενο ενώ τα φύλλα 
ένα μικρό υποσύνολο του.

Η επιλογή του τύπου οριοθετικού όγκου για μια δεδομένη εφαρμογή 
καθορίζεται από διάφορους παράγοντες. Τέτοιοι είναι το κόστος υπολογισμού
ενός οριοθετικού όγκου για ένα αντικείμενο, το κόστος της ενημέρωσης
του σε εφαρμογές στις οποίες τα αντικείμενα μπορούν να μετακινηθούν 
ή να αλλάξουν σχήμα, το κόστος ανίχνευσης σύγκρουσης 
ή υπολογισμού απόστασης και η επιθυμητή 
ακρίβεια για ελέγχους σύγκρουσης ή απόστασης. Η ακρίβεια αυτή σχετίζεται 
με τον όγκο του \textit{κενού χώρου} που περικλείεται από τον οριοθετικό όγκο
όμως δε σχετίζεται με το οριοθετημένο αντικείμενο. Τυπικά, ισχύει ο εξής
συμβιβασμός: οι πιο εκλεπτυσμένοι οριοθετικοί όγκοι περικλείουν γενικά 
λιγότερο κενό χώρο, αλλά είναι πιο ακριβοί υπολογιστικά. 

Οι πιο συνηθισμένοι τύποι οριοθετικών όγκων είναι:
\begin{itemize}
    \item Η \textbf{Οριοθετική Σφαίρα (\tl{Bounding Sphere})}, 
    η οποία είναι μια σφαίρα που περικλείει το αντικείμενο. 
    Αναπαρίσταται από το κέντρο και την ακτίνα της και 
    επιτρέπει πολύ γρήγορους ελέγχους σύγκρουσης και 
    υπολογισμού απόστασης. 
    Δύο σφαίρες τέμνονται όταν η απόσταση μεταξύ των κέντρων τους 
    δεν υπερβαίνει το άθροισμα των ακτίνων τους.

    \item Ο \textbf{Οριοθετικός Κύλινδρος (\tl{Bounding Cylinder})}, 
    είναι ένας κύλινδρος που περικλείει το αντικείμενο. 
    Στις περισσότερες εφαρμογές ο άξονας του κυλίνδρου είναι 
    ευθυγραμμισμένος με την κατακόρυφη διεύθυνση της σκηνής.
    Οι κύλινδροι είναι κατάλληλοι για τρισδιάστατα αντικείμενα 
    που μπορούν να περιστρέφονται μόνο γύρω από έναν κατακόρυφο άξονα 
    αλλά όχι γύρω από άλλους άξονες, και κατά τα άλλα η κίνηση τους 
    να είναι μόνο μεταφορική. 
    Δύο κύλινδροι ευθυγραμμισμένοι με κατακόρυφο άξονα τέμνονται όταν, 
    ταυτόχρονα, τέμνονται οι προβολές τους στον κατακόρυφο άξονα (που είναι 
    ευθύγραμμα τμήματα), καθώς και οι προβολές τους 
    στο οριζόντιο επίπεδο (που είναι κύκλοι). 
    Οι δύο συνθήκες είναι εύκολο να ελεγχθούν. 
    Στα βιντεοπαιχνίδια, οι οριοθετικοί κύλινδροι χρησιμοποιούνται 
    συχνά ως οριοθετικοί όγκοι για χαρακτήρες που στέκονται όρθια.

    \item Το \textbf{Οριοθετικό Πλαίσιο (\tl{Bounding Box})}, είναι ένα 
    ορθογώνιο παραλληλεπίπεδο που περικλείει το αντικείμενο. 
    Στις προσομοιώσεις όπου η σκηνή αλλάζει δυναμικά, τα οριοθετικά πλαίσια
    προτιμώνται από άλλα σχήματα οριοθετικών όγκων (σφαίρες ή κυλίνδρους) 
    για αντικείμενα που έχουν χονδρικά κυβοειδές σχήμα
    όταν ο έλεγχος σύγκρουσης πρέπει να είναι αρκετά ακριβής.
    Το όφελος είναι προφανές, για παράδειγμα, για αντικείμενα 
    που ακουμπούν πάνω σε άλλα, όπως ένα αυτοκίνητο που ακουμπά 
    στο έδαφος: μια οριοθετική σφαίρα θα έδειχνε πως το αυτοκίνητο 
    πιθανώς να τέμνεται με το έδαφος, το οποίο στη συνέχεια θα έπρεπε 
    να απορριφθεί από μια πιο ακριβή υπολογιστικά δοκιμή του πραγματικού 
    μοντέλου του αυτοκινήτου. Ένα οριοθετικό πλαίσιο δείχνει αμέσως 
    ότι το αυτοκίνητο δεν τέμνεται με το έδαφος, 
    εξοικονομώντας έτσι την κοστοβόρα δοκιμή.

    Στη γενική περίπτωση, 
    ένα αυθαίρετο οριοθετικό πλαίσιο ονομάζεται και \textbf{Προσανατολισμένο 
    Οριοθετικό Πλαίσιο (\tl{Oriented Bounding Box})} \textbf{\tl{OBB}} 
    ή \textbf{\tl{OOBB}} όταν χρησιμοποιείται το τοπικό σύστημα 
    συντεταγμένων ενός αντικειμένου.
    Σε πολλές εφαρμογές το οριοθετικό πλαίσιο είναι ευθυγραμμισμένο με τους 
    άξονες του συστήματος συντεταγμένων και ονομάζεται \textbf{Οριοθετικό 
    Πλαίσιο Ευθυγραμμισμένο με τους Άξονες (\tl{Axis-Aligned Bounding Box})}
    ή \textbf{\tl{AABB}}. 
    Τα \tl{AABB} είναι πιο απλά και αποδοτικά στην ανίχνευση 
    σύγκρουσης μεταξύ τους από τα \tl{OBB}, αλλά έχουν το μειονέκτημα ότι 
    όταν το μοντέλο  περιστρέφεται δεν μπορούν απλώς να περιστραφούν με αυτό, 
    αλλά πρέπει να υπολογιστούν εκ νέου. 

    Στην περίπτωση των δύο διαστάσεων, χρησιμοποιείται το 
    \textbf{Ελάχιστο Οριοθετικό Παραλληλόγραμμο (\tl{Minimum Bounding Rectangle})}
    ή \tl{\textbf{MBR}}. Το \tl{MBR} είναι ειδική περίπτωση του \tl{AABB} στο 
    επίπεδο και χρησιμοποιείται συχνά για να περικλείει γεωγραφικά (ή γεωχωρικά)
    δεδομένα.

    \item H \textbf{Οριοθετική Κάψουλα (\tl{Bounding Capsule})}, 
    η οποία προκύπτει από τον συνολικό όγκο που περικλείει μια σφαίρα 
    καθώς κινείται πάνω σε ένα ευθύγραμμο τμήμα (\tl{swept sphere}). 
    Ο όγκος που προκύπτει αποτελείται από έναν κύλινδρο και δύο 
    ημισφαίρια στα άκρα του.
    Μπορούν να αναπαρασταθούν από το ευθύγραμμο τμήμα και το μήκος 
    της ακτίνας της σφαίρας.
    Έχει χαρακτηριστικά παρόμοια με έναν κύλινδρο, αλλά είναι πιο εύκολη 
    στη χρήση, επειδή ο έλεγχος σύγκρουσης είναι απλούστερος.
    Για παράδειγμα, δύο κάψουλες τέμνονται εάν η απόσταση μεταξύ των τμημάτων
    που τις ορίζουν είναι μικρότερη από το άθροισμα των ακτίνων τους.
    Αυτό ισχύει για τις αυθαίρετα προσανατολισμένες κάψουλες, γι' 
    αυτό είναι πιο ελκυστικές από τους κυλίνδρους στην πράξη.
\end{itemize}

Στα πλαίσια αυτής της εργασίας, θα χρησιμοποιηθούν τα \tl{AABB}.

\section{Ιεραρχίες Οριοθετικών Όγκων}

\label{sec:bvh}

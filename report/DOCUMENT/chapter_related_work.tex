\chapter{Σχετική Βιβλιογραφία}
\label{ch:related_work}
Τα ερωτήματα εγγύτητας και η ανίχνευση σύγκρουσης διερευνώνται 
εκτενώς για δεκαετίες από ερευνητές στα γραφικά υπολογιστών, στη ρομποτική,
στις προσομοίωσες, την υπολογιστική γεωμετρία και τα κινούμενα σχέδια 
υπολογιστών. 
Εκτός από την ανίχνευση σύγκρουσης και τον υπολογισμό απόστασης, υπάρχει
σημαντική βιβλιογραφία σχετική με τις Ιεραρχίες Οριοθετικών Όγκων 
(\tl{BVH}) και τις διάφορες εφαρμογές τους. 
Σε αυτό το κεφάλαιο κάνουμε αναφορά σε άρθρα της βιβλιογραφίας 
που μελετούν το ίδιο ή παρόμοιο πρόβλημα με το δικό μας
και επισημαίνουμε τις τεχνικές που χρησιμοποιούνται 
και στη δική μας μεθοδολογία.

\section{Εύρεση Κοντινότερου Σημείου σε ένα Σύνολο Σημείων}

\section{Ανίχνευση Σύγκρουσης και Υπολογισμός Απόστασης Πολυγώνων}
% να πω για linear programming 
% These include Dobkin-
% Kirkpatrick hierarchies [DK82], linear programming [Sei90] and algorithms for intersecting
% convex p olytop es [Cha89 ] 
% και γενικά ό,τι αναφέρεται στο paper του GJK.

\section{Ιεραρχικές Δομές Δεδομένων}
%να αναφέρω SAH
%R-Trees
%BSP Trees
% Απόσταση Σημείου από Πολυγωνικό Πλέγμα
% Απόσταση Δύο Πολυγωνικών Πλεγμάτων
% Απόσταση Αντικειμένων που Περιγράφονται από \tl{NURBS}
